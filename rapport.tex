\documentclass[titlepage]{article}
\usepackage[utf8]{inputenc}
\usepackage[francais]{babel}
\usepackage[T1]{fontenc}
\usepackage{graphicx}
\usepackage{caption} 
\usepackage{listings}
\usepackage{xcolor}
\usepackage{tabularx}
\usepackage{colortbl}
\usepackage{mathbbol}
\usepackage{amssymb}
\usepackage{amsfonts}
\usepackage{amsmath}


\title{Rapport MT10 - TP3 : Corps finis et corps correcteurs}
\author{Océane Bordeau, Martin Schneider}
\date{10 mai 2022}

\setlength{\parindent}{0pt}
\definecolor{codegreen}{rgb}{0,0.6,0}
\definecolor{codegray}{rgb}{0.5,0.5,0.5}
\definecolor{gray}{rgb}{0.8,0.8,0.8}
\definecolor{codepurple}{rgb}{0.58,0,0.82}
\definecolor{codeblue}{rgb}{0,0,255}
\definecolor{backcolour}{rgb}{0.95,0.95,0.92}

\lstdefinestyle{mystyle}{ 
    commentstyle=\color{magenta},
    keywordstyle=\color{codeblue},
    numberstyle=\tiny\color{codegray},
    stringstyle=\color{codepurple},
    basicstyle=\ttfamily\footnotesize,
    breakatwhitespace=false,         
    breaklines=true,                 
    captionpos=b,                    
    keepspaces=true,                                  
    showspaces=false,                
    showstringspaces=false,
    showtabs=false,                  
    tabsize=2
}

\lstset{style=mystyle}

\begin{document}
    \maketitle
    \tableofcontents
    \pagebreak

    \section{Construction de corps finis}
        \setcounter{subsection}{2}
        \subsection{Dénombrement des polynômes irréductibles et unitaires de $\mathbb{F}_p[X]$}
            \subsubsection{Factorisation de $X^q-X$ dans $\mathbb{F}_p[X]$}
            \textbf{Question 1 :}

            L'objetcif est de vérifier la factorisation du pôlynome $X^q-X$ avec $q=2^6$.

            Pour cela, nous allons déterminer l'ensemble des pôlynomes unitaires et irréductibales de degré $d$ tels que $d|n$, puis effectuer leur produit.

            \lstinputlisting[language=Python, firstline=79, lastline=88]{utils.sage.py}

            \begin{tabularx}{12cm}{|p{0.60cm}|X|}
                \hline
                \rowcolor{gray}
                \texttt{In}
                & 
                \texttt{prodduit == R(x ** (p ** n) - x)}
                \\
                \hline
                \texttt{Out}
                &
                \texttt{True}
                \\
                \hline
            \end{tabularx}
            \bigbreak

            On peut donc affirmer :

            \[ X^q-X=\prod_{\substack{P \in \mathbb{F}_p[X] \\ irr\acute{e}ductible \: et \: unitaire  \\ deg(P)|n}} P(X) \]

            \subsubsection{La fonction de Möbius}
            \textbf{Question 2 :}

            \lstinputlisting[language=Python, firstline=89, lastline=93]{utils.sage.py}

            \begin{tabularx}{12cm}{|p{0.60cm}|X|}
                \hline
                \rowcolor{gray}
                \texttt{In}
                & 
                \texttt{test1(100)}
                \\
                \hline
                \texttt{Out}
                &
                \texttt{True}
                \\
                \hline
            \end{tabularx}
            \bigbreak

            On vérifie donc que $\mu (n) \in \{-1;0;1\}$ pour les 100 premiers entiers naturels.

            \lstinputlisting[language=Python, firstline=95, lastline=99]{utils.sage.py}

            \begin{tabularx}{12cm}{|p{0.60cm}|X|}
                \hline
                \rowcolor{gray}
                \texttt{In}
                & 
                \texttt{test2(100)}
                \\
                \hline
                \texttt{Out}
                &
                \texttt{True}
                \\
                \hline
            \end{tabularx}
            \bigbreak

            La formule d'Euler est vérifiée pour les 100 premiers entiers naturels.

            \lstinputlisting[language=Python, firstline=101, lastline=105]{utils.sage.py}

            \begin{tabularx}{12cm}{|p{0.60cm}|X|}
                \hline
                \rowcolor{gray}
                \texttt{In}
                & 
                \texttt{phiMobius(100)}
                \\
                \hline
                \texttt{Out}
                &
                \texttt{40}
                \\
                \hline
            \end{tabularx}
            \bigbreak

            \begin{tabularx}{12cm}{|p{0.60cm}|X|}
                \hline
                \rowcolor{gray}
                \texttt{In}
                & 
                \texttt{euler\_phi(100)}
                \\
                \hline
                \texttt{Out}
                &
                \texttt{40}
                \\
                \hline
            \end{tabularx}
            \bigbreak

            Le calcul de $\phi (100)$ grâce à la formule d'inversion de Möbius donne bien le bon résultat : $40$.

            \subsubsection{Calcul du nombre de polynômes unitaires irréductibles de degré $d$ dans $\mathbb{F}_p[X]$}
            \textbf{Question 3 :}
            \lstinputlisting[language=Python, firstline=126, lastline=130]{utils.sage.py}
            Grâce à cette fonction, dressons un tableau de $Irr_p(n)$ pour $p = 2, 3, 5$ et $n = 1, . . . , 10$.

            \begin{tabular}{|c|c c c c c c c c c c|}
                \hline
                & 1 & 2 & 3 & 4 & 5 & 6 & 7 & 8 & 9 & 10 \\
                \hline
                2 & 2 & 1 & 2 & 3 & 6 & 9 & 18 & 30 & 56 & 99 \\
                3 & 3 & 3 & 8 & 18 & 48 & 116 & 312 & 810 & 2184 & 5880 \\
                5 & 5 & 10 & 40 & 150 & 624 & 2580 & 11160 &  48750 & 217000 & 976248 \\ 
                \hline
            \end{tabular}
            \bigbreak

            \textbf{Question 4 :}
            \lstinputlisting[language=Python, firstline=107, lastline=124]{utils.sage.py}

            \begin{tabularx}{12cm}{|p{0.60cm}|X|}
                \hline
                \rowcolor{gray}
                \texttt{In}
                & 
                \texttt{polynomes10()}
                \\
                \hline
                \texttt{Out}
                &
                \texttt{226}
                \\
                \hline
            \end{tabularx}
            \bigbreak

        \subsection{Calcul de polynômes unitaires irréductibles de $\mathbb{F}_p[X]$}
    
    \section{Les codes de Reed et Solomon}
        \subsection{Définition des codes de Reed-Solomon \\ généralisés (GRS)}

        \textbf{Question 5 :}
        La fonction \texttt{codeGRS} prend en entrée un bloc d'entiers $message$ de longueur $k$, et les paramètres $v$ et $\alpha$ de longueur $n$.
        On associera chaque entier de $message$ avec un élément dans $\mathbb{F}_q$. Les entiers doivent donc être inférieurs à $q$.
        Par définition, on a $0 \leqslant k \leqslant n \leqslant q$. La fonction retourne $y = ev_{\alpha,v}(f) = (v_0f(\alpha_0), v_1f(\alpha_1), ..., v_{n-1}f(\alpha_{n-1}))$.

        Les paramètres $v$ et $\alpha$ sont des listes d'éléments tirés au hasard dans $\mathbb{F}_q$.

        \lstinputlisting[language=Python, firstline=1, lastline=8]{utils.sage.py}

        \begin{tabularx}{12cm}{|p{0.60cm}|X|}
            \hline
            \rowcolor{gray}
            \texttt{In}
            & 
            \texttt{q = 2**8\newline
            Fq = GF(q, name='a')\newline
            R.<X> = Fq['X']\newline
            C = Fq.list()\newline
            n, k = 8, 3\newline
            \newline
            x = [254, 20, 11]\newline
            \newline
            V = []\newline
            for i in range(n):\newline
            V.append(C[randint(1, len(C)-1)])\newline
            \newline
            A = []\newline
            for i in range(n): \newline
            \_\_ c = C[randint(1, len(C)-1)]\newline
            \_\_ while c in A:
            c = C[randint(1, len(C)-1)]\newline
            a.append(c)}
            \\
            \hline
            \rowcolor{gray}
            \texttt{In}
            & 
            \texttt{y = codeGRS(x, V, A)\newline
            y}
            \\
            \hline
            \texttt{Out}
            &
            \texttt{$[a^6 + a^4 + a^2,\newline
            a^5 + a^4 + a^3 + a^2 + a + 1,\newline
            a^5 + a^3,\newline
            a^5 + a^4 + a + 1,\newline
            a^6 + a^3 + a^2,\newline
            a^7 + a^5 + a^4 + a^3,\newline
            a^5,\newline
            a^7 + a^4 + a^3 + a + 1]$}
            \\
            \hline
        \end{tabularx}
        \bigbreak

        \subsection{Cas sans erreur : décodage des GRS par interpolation de Lagrange}
        \textbf{Question 6 :}
        Grâce à l'interpolation de Lagrange qui permet de reconstruire un polynome avec plusieurs valeurs de $x$ et $y$ données. 
        On peut retrouver le message du départ à partir du message encodé.

        Le polynome est défini par :
        \[\sum_{i=0}^{n-1}y_il_i(X)\]
        avec :
        \[l_i(X) = \prod_{j = 0; j \neq i}^{n - 1}\frac{X-x_j}{x_i-x_j} \]

        Ici on trouve $y_j$ grâce au message encodé, en effet : 
        \[y_i = f(\alpha_i) = \frac{v_i}{messageEncode[i]}\]

        \lstinputlisting[language=Python, firstline=10, lastline=26]{utils.sage.py}

        \begin{tabularx}{12cm}{|p{0.60cm}|X|}
            \hline
            \rowcolor{gray}
            \texttt{In}
            & 
            \texttt{decodeGRS(y, V, A)}
            \\
            \hline
            \texttt{Out}
            &
            \texttt{[254, 20, 11]}
            \\
            \hline
        \end{tabularx}
        \bigbreak

        Si on ne choisit pas les point $\alpha_i$ distincts deux à deux, on risque de diviser par 0 à un moment donné dans le polynôme de Lagrange.

        \subsection{Simulation d’erreurs de transmission}
        \textbf{Question 7 :}
        Chaque mot codé $y$ est envoyé par un canal de transmission, il se peut que d'éventuelles erreurs font que nous recevons un mot codé $y'$ différent.
        On cherche à simuler des erreurs de transmission avec la fonction \texttt{errTrans} qui prend en paramètre $y$ le mot codé, et $Nb\_err$ le nombre d'erreurs de transmissions.
        On tire aléatoirement chaque erreur dans $\mathbb{F}_q^*$. La fonction retourne $y'$ le mot codé avec $Nb\_err$ de transmission.

        \lstinputlisting[language=Python, firstline=28, lastline=34]{utils.sage.py}

        \begin{tabularx}{12cm}{|p{0.60cm}|X|}
            \hline
            \rowcolor{gray}
            \texttt{In}
            & 
            \texttt{yp = errTrans(y, 2)\newline
            y\newline
            yp}
            \\
            \hline
            \texttt{Out}
            &
            \texttt{$[a^6 + a^4 + a^2,\newline
            a^5 + a^4 + a^3 + a^2 + a + 1,\newline
            a^5 + a^3,\newline
            a^5 + a^4 + a + 1,\newline
            a^6 + a^3 + a^2,\newline
            a^7 + a^5 + a^4 + a^3,\newline
            a^5,\newline
            a^7 + a^4 + a^3 + a + 1]$\newline\newline
            $[a^6 + a^4 + a^2,\newline
            a^5 + a^4 + a^3 + a^2 + a + 1,\newline
            a^5 + a^2 + a,\newline
            a^7 + a^6 + a^5 + a^3 + a + 1,\newline
            a^6 + a^3 + a^2,\newline
            a^6 + a^5 + a^4 + a^3 + a^2 + 1,\newline
            a^5,\newline
            a^7 + a^4 + a^3 + a + 1]$}
            \\
            \hline
        \end{tabularx}
        \bigbreak

        \textbf{Question 8 :}
        On vérifie que l'interpolation de Lagrange donne n'importe quoi dès qu'il y a une erreur de transmission.

        \begin{tabularx}{12cm}{|p{0.60cm}|X|}
            \hline
            \rowcolor{gray}
            \texttt{In}
            & 
            \texttt{decodeGRS(y, V, A)}
            \\
            \hline
            \texttt{Out}
            &
            \texttt{[254, 20, 11]}
            \\
            \hline
            \rowcolor{gray}
            \texttt{In}
            & 
            \texttt{decodeGRS(yp, V, A)}
            \\
            \hline
            \texttt{Out}
            &
            \texttt{[168, 245, 225, 26, 148, 173, 42, 243]}
            \\
            \hline
        \end{tabularx}
        \bigbreak

    \section{Correction d’erreurs grâce aux GRS}
        \subsection{Le polynôme syndrome}
        \textbf{Question 9 :}
        On peut calculer le polynôme syndrome avec les paramètres $y'$ le mot reçu, les vecteurs $v$ et $\alpha$.
        On définit $r$ par $r = n - k$.

        Le polynome syndrome est définit par : 
        \[S_{y'}(X) = \sum_{i = 0}^{n - 1}y'_i(v_i^{-1}L_i(\alpha_i)^{-1})(\sum_{j = 0}^{r - 1}(\alpha_iX)^j)\]

        Si $S(X) = 0$, on a donc $y' \in GRS_{n,k}(\alpha,v)$ et on a alors de grandes chances pour que $y'$ ne contienne pas d'erreurs.

        \lstinputlisting[language=Python, firstline=36, lastline=41]{utils.sage.py}

        \textbf{Question 10 :}\bigbreak

        \begin{tabularx}{12cm}{|p{0.60cm}|X|}
            \hline
            \rowcolor{gray}
            \texttt{In}
            & 
            \texttt{print(f"S(X) = \{Syndrome(y, A, V)\}")}
            \\
            \hline
            \texttt{Out}
            &
            \texttt{$S(X) = 0$}
            \\
            \hline
            \rowcolor{gray}
            \texttt{In}
            & 
            \texttt{print(f"S(X) = \{Syndrome(yp, A, V)\}")}
            \\
            \hline
            \texttt{Out}
            &
            \texttt{$S(X) = (a^4 + a^2 + a + 1)*X^4 + (a^6 + a^2 + a + 1)*X^3 + (a^5 + a^4 + a^3 + a^2 + a)*X^2 + a^3*X + a^6 + a^5 + a^4 + a^3 + a^2 + 1$}
            \\
            \hline
        \end{tabularx}
        \bigbreak

        On a bien $y' \in C \Longleftrightarrow S(X) = 0$.

        \subsection{L’équation clef}
        L'équation clef est la suivante : 
        \[\sigma(X)S(X) = \omega(X) [X^r]\]
        avec le polynôme localisateur des erreurs $\sigma(X)$ et le polynôme évaluateur des erreurs $\omega(X)$.

        \subsection{Résolution de l’équation clef par Euclide}
        
        \textbf{Question 11 :}
        Grâce à l'algorithme d'Euclide on peut trouver $\sigma$ et $\omega$, la fonction \texttt{Clef} 
        calcule à partir du polynôme syndrome $S(X)$, de $q$, $k$ et $n$, les polynômes $\sigma$ et $\omega$.

        \lstinputlisting[language=Python, firstline=43, lastline=57]{utils.sage.py}
        
        Avec l'exemple développé jusqu'ici on vérifie que l'équation clef est satisfaite.

        \begin{tabularx}{12cm}{|p{0.60cm}|X|}
            \hline
            \rowcolor{gray}
            \texttt{In}
            & 
            \texttt{r = n - k\newline
            S = Syndrome(yp, a, v)\newline
            sigma, omega = Clef(S)\newline
            print(sigma)
            print(omega)}
            \\
            \hline
            \texttt{Out}
            &
            \texttt{$(a^7 + a^6 + a^5 + a^3 + a)*X^2 + (a^6 + a^5 + a^4 + a^3 + a^2 + 1)*X + 1\newline\newline
            (a^7 + a^2 + 1)*X^2 + (a^7 + a^6 + a^5 + a^4 + a^3 + a^2 + 1)*X + a^6 + a^5 + a^4 + a^3 + a^2 + 1$}
            \\
            \hline
            \rowcolor{gray}
            \texttt{In}
            & 
            \texttt{print(((sigma * R(S)) \% X**r) == (omega\% X**r))}
            \\
            \hline
            \texttt{Out}
            &
            \texttt{True}
            \\
            \hline
        \end{tabularx}
        \bigbreak

        \subsection{Localisation et évaluation des erreurs de transmission}
        \textbf{Question 12 :}
        La fonction \texttt{Erreur} à partir des polynômes localisateur et évaluateur $\sigma$ et $\omega$, et des vecteurs $v$ et $\alpha$,
        retourne un tableau $e$ correspondant à l'erreur de transmission.

        On peut calculer $e_b$ de la façon suivante : 
        \[e_b = -\alpha_b*\omega(\alpha_b^{-1})*v_b*L_b(\alpha_b)*(\sigma'(\alpha_b^{-1}))^{-1}\]
        en sachant que : 
        \[B = \{b \in [0,n-1] : \sigma(\alpha_b^{-1}) = 0 \}\]

        \lstinputlisting[language=Python, firstline=59, lastline=67]{utils.sage.py}

        \begin{tabularx}{12cm}{|p{0.60cm}|X|}
            \hline
            \rowcolor{gray}
            \texttt{In}
            & 
            \texttt{e = Erreur(sigma, omega, A, V)\newline
            yc = [i-j for i, j in zip(yp, e)]\newline
            decodeGRS(yc, V, A)}
            \\
            \hline
            \texttt{Out}
            &
            \texttt{[254, 20, 11]}
            \\
            \hline
        \end{tabularx}
        \bigbreak

    \section{Conclusion : une chaîne de transmission cryptée robuste}
    L'objectif de cette section est de simuler un chaîne de transmission qui allie le l'algorithme RSA pour l'encryption et le code GRS pour l'intégrité des données.
    Dans un soucis de simplification, on utilisera une clé RSA de 8 bits pour obtenir une liste d'entiers compris entre 0 et 255. Cela facilitera l'utilisation d'un corps fini à $2^8$ éléments.
    Les fonctions d'encryption et de décryption utilisée provienent du précédent TP.
    
    Pour obtenir des mots de k lettres, on utilisera la fonction \texttt{split} qui divise la chaine d'entiers en une liste de sous-listes contenant k éléments. 
    Cette dernière liste pourra donc être ensuite codée grâce à la fonction \texttt{codeGRS}.

    \lstinputlisting[language=Python, firstline=132, lastline=150]{utils.sage.py}

        \subsection{Cas sans erreur}

        Le message doit d'abord être numérisé sous forme de liste d'entiers. Puis cette dernière liste est encryptée grâce à la fonction \texttt{encodeRSA}. Cette liste est ensuite divisée en mots de $k$ lettres pour pouvoir être encodée.
        \bigbreak

        \begin{tabularx}{12cm}{|p{0.60cm}|X|}
            \hline
            \rowcolor{gray}
            \texttt{In}
            & 
            \texttt{cryted\_message = encodeRSA(numerise("Voici le message à crypter !", 24), e , N)}
            \\
            \hline
            \rowcolor{gray}
            \texttt{In}
            &
            \texttt{grs\_encoded = [codeGRS(word, v, a) for word in split(crypted\_message)]}
            \\
            \hline
            \rowcolor{gray}
            \texttt{In}
            &
            \texttt{gathered\_message = gather(decodeGRS(word, v, a) for word in grs\_encoded])}
            \\
            \hline
            \rowcolor{gray}
            \texttt{In}
            &
            \texttt{alphabetise\_safe(decode\_rsa(gathered\_message, d, N), 24)}
            \\
            \hline
            \texttt{Out}
            &
            \texttt{Voici le message à crypter !}
            \\
            \hline
        \end{tabularx}
        \bigbreak
    
    \subsection{Cas avec erreur}

        Le processus est le même, simule seulement l'apparition d'erreurs lors de la transmission du message.

        \begin{tabularx}{12cm}{|p{0.60cm}|X|}
            \hline
            \rowcolor{gray}
            \texttt{In}
            & 
            \texttt{cryted\_message = encodeRSA(numerise("Voici le message à crypter !", 24), e , N)}
            \\
            \hline
            \rowcolor{gray}
            \texttt{In}
            &
            \texttt{grs\_encoded = [codeGRS(word, v, a) for word in split(crypted\_message)]}
            \\
            \hline
            \rowcolor{gray}
            \texttt{In}
            &
            \texttt{gathered\_message = gather(decodeGRS(word, v, a) for word in grs\_encoded])}
            \\
            \hline
            \rowcolor{gray}
            \texttt{In}
            &
            \texttt{alphabetise\_safe(decode\_rsa(gathered\_message, d, N), 24)}
            \\
            \hline
            \texttt{Out}
            &
            \texttt{Voici le message à crypter !}
            \\
            \hline
        \end{tabularx}
        \bigbreak
        
\end{document}