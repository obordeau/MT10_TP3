\documentclass[titlepage]{article}
\usepackage[utf8]{inputenc}
\usepackage[T1]{fontenc}
\usepackage{graphicx}
\usepackage{caption} 
\usepackage{listings}
\usepackage{xcolor}
\usepackage{tabularx}
\usepackage{colortbl}
\usepackage{mathbbol}
\usepackage{amssymb}

\title{Rapport MT10 - TP3 : Corps finis et corps correcteurs}
\author{Océane Bordeau, Martin Schneider}
\date{10 mai 2022}

\setlength{\parindent}{0pt}
\definecolor{codegreen}{rgb}{0,0.6,0}
\definecolor{codegray}{rgb}{0.5,0.5,0.5}
\definecolor{gray}{rgb}{0.8,0.8,0.8}
\definecolor{codepurple}{rgb}{0.58,0,0.82}
\definecolor{codeblue}{rgb}{0,0,255}
\definecolor{backcolour}{rgb}{0.95,0.95,0.92}

\lstdefinestyle{mystyle}{ 
    commentstyle=\color{magenta},
    keywordstyle=\color{codeblue},
    numberstyle=\tiny\color{codegray},
    stringstyle=\color{codepurple},
    basicstyle=\ttfamily\footnotesize,
    breakatwhitespace=false,         
    breaklines=true,                 
    captionpos=b,                    
    keepspaces=true,                                  
    showspaces=false,                
    showstringspaces=false,
    showtabs=false,                  
    tabsize=2
}

\lstset{style=mystyle}

\begin{document}
    \maketitle
    \tableofcontents
    \pagebreak

    \section{Construction de corps finis}
        \setcounter{subsection}{2}
        \subsection{Dénombrement des polynômes irréductibles et unitaires de $\mathbb{F}_p[X]$}
            \subsubsection{Factorisation de $X^q-X$ dans $\mathbb{F}_p[X]$}
            Question 1
            \subsubsection{La fonction de Möbius}
            Question 2

            1.
            % \lstinputlisting[language=Python, firstline=1, lastline=5]{utils.sage.py}
            2.
            % \lstinputlisting[language=Python, firstline=7, lastline=17]{utils.sage.py}
            3.
            % \lstinputlisting[language=Python, firstline=19, lastline=25]{utils.sage.py}
            \subsubsection{Calcul du nombre de polynômes unitaires irréductibles de degré $d$ dans $\mathbb{F}_p[X]$}
            Question 3
            % \lstinputlisting[language=Python, firstline=27, lastline=33]{utils.sage.py}
            Question 4
            % \lstinputlisting[language=Python, firstline=35, lastline=45]{utils.sage.py}
        \subsection{Calcul de polynômes unitaires irréductibles de $\mathbb{F}_p[X]$}
    
    \section{Les codes de Reed et Solomon}

        \subsection{Définition des codes de Reed-Solomon généralisés (GRS)}

        \textbf{Question 5 :}
        La fonction \texttt{codeGRS} prend en entrée $q$, un bloc de message $x$ de longueur $k$, et les paramètre $v$ et $\alpha$ de longueur $n$.
        Avec $0 \leqslant k \leqslant n \leqslant q$. La fonction retourne $y = ev_{\alpha,v}(f) = (v_0f(\alpha_0), v_1f(\alpha_1), ..., v_{n-1}f(\alpha_{n-1}))$.

        \lstinputlisting[language=Python, firstline=1, lastline=21]{utils.sage.py}

        \begin{tabularx}{12cm}{|p{0.60cm}|X|}
            \hline
            \rowcolor{gray}
            \texttt{In}
            & 
            \texttt{message = [2, 8, 6, 4, 9, 10, 7] \newline
            V = [25, 41, 44, 27, 10, 24, 37, 21, 3, 13] \newline
A =         [41, 7, 6, 49, 3, 1, 9, 19, 18, 35] \newline
            messageEncode = codeGRS(211, message, V, A) \newline
            messageEncode}
            \\
            \hline
            \texttt{Out}
            &
            \texttt{[41, 120, 201, 33, 70, 147, 135, 80, 9, 166]}
            \\
            \hline
        \end{tabularx}
        \bigbreak

        \subsection{Cas sans erreur : décodage des GRS par interpolation de Lagrange}
        Grâce à l'interpolation de Lagrange qui permet de reconstruire un polynome avec plusieurs valeurs de $x$ et $y$ données. 
        On peut retrouver le message du départ à partir du message encodé.

        Le polynome est défini par :
        \[\sum_{i=0}^{n-1}y_il_i(X)\]
        avec :
        \[l_i(X) = \prod_{j = 0; j \neq i}^{n - 1}\frac{X-x_j}{x_i-x_j} \]

        Ici on trouve $y_j$ grâce au message encodé, en effet : 
        \[y_i = f(\alpha_i) = \frac{v_i}{messageEncode[i]}\]

        \lstinputlisting[language=Python, firstline=23, lastline=55]{utils.sage.py}

        \begin{tabularx}{12cm}{|p{0.60cm}|X|}
            \hline
            \rowcolor{gray}
            \texttt{In}
            & 
            \texttt{decodeGRS(211, messageEncode, V, A)}
            \\
            \hline
            \texttt{Out}
            &
            \texttt{[2, 8, 6, 4, 9, 10, 7]}
            \\
            \hline
        \end{tabularx}
        \bigbreak

        Si on ne choisit pas les point $\alpha_i$ distincts deux à deux, on risque de diviser par 0 à un moment donné dans le polynôme de Lagrange.

        \subsection{Simulation d’erreurs de transmission}


    \section{Correction d’erreurs grâce aux GRS}
        \subsection{Le polynôme syndrome}
        \subsection{L’équation clef}
        \subsection{Résolution de l’équation clef par Euclide}
        \subsection{Localisation et évaluation des erreurs de transmission}
    \section{Conclusion : une chaîne de transmission cryptée robuste}

\end{document}