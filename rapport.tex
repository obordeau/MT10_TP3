\documentclass[titlepage]{article}
\usepackage[utf8]{inputenc}
\usepackage[T1]{fontenc}
\usepackage{graphicx}
\usepackage{caption} 
\usepackage{listings}
\usepackage{xcolor}
\usepackage{tabularx}
\usepackage{colortbl}
\usepackage{mathbbol}

\title{Rapport MT10 - TP3 : Corps finis et corps correcteurs}
\author{Océane Bordeau, Martin Schneider}
\date{10 mai 2022}

\setlength{\parindent}{0pt}
\definecolor{codegreen}{rgb}{0,0.6,0}
\definecolor{codegray}{rgb}{0.5,0.5,0.5}
\definecolor{gray}{rgb}{0.8,0.8,0.8}
\definecolor{codepurple}{rgb}{0.58,0,0.82}
\definecolor{codeblue}{rgb}{0,0,255}
\definecolor{backcolour}{rgb}{0.95,0.95,0.92}

\lstdefinestyle{mystyle}{ 
    commentstyle=\color{magenta},
    keywordstyle=\color{codeblue},
    numberstyle=\tiny\color{codegray},
    stringstyle=\color{codepurple},
    basicstyle=\ttfamily\footnotesize,
    breakatwhitespace=false,         
    breaklines=true,                 
    captionpos=b,                    
    keepspaces=true,                                  
    showspaces=false,                
    showstringspaces=false,
    showtabs=false,                  
    tabsize=2
}

\lstset{style=mystyle}

\begin{document}
    \maketitle
    \tableofcontents
    \pagebreak
    \section{Construction de corps finis}
        \setcounter{subsection}{2}
        \subsection{Dénombrement des polynômes irréductibles et unitaires de $\mathbb{F}_p[X]$}
            \subsubsection{Factorisation de $X^q-X$ dans $\mathbb{F}_p[X]$}
            Question 1
            \subsubsection{La fonction de Möbius}
            Question 2

            1.
            \lstinputlisting[language=Python, firstline=1, lastline=5]{utils.sage}
            2.
            \lstinputlisting[language=Python, firstline=7, lastline=17]{utils.sage}
            3.
            \lstinputlisting[language=Python, firstline=19, lastline=25]{utils.sage}
            \subsubsection{Calcul du nombre de polynômes unitaires irréductibles de degré $d$ dans $\mathbb{F}_p[X]$}
            Question 3
            \lstinputlisting[language=Python, firstline=27, lastline=33]{utils.sage}
            Question 4
            \lstinputlisting[language=Python, firstline=35, lastline=45]{utils.sage}
        \subsection{Calcul de polynômes unitaires irréductibles de $\mathbb{F}_p[X]$}
    \section{Les codes de Reed et Solomon}
        \subsection{Définition des codes de Reed-Solomon généralisés (GRS)}
        Question 5
        \lstinputlisting[language=Python, firstline=48, lastline=65]{utils.sage}
        \subsection{Cas sans erreur : décodage des GRS par interpolation de Lagrange}
        \subsection{Simulation d’erreurs de transmission}
    \section{Correction d’erreurs grâce aux GRS}
        \subsection{Le polynôme syndrome}
        \subsection{L’équation clef}
        \subsection{Résolution de l’équation clef par Euclide}
        \subsection{Localisation et évaluation des erreurs de transmission}
    \section{Conclusion : une chaîne de transmission cryptée robuste}

\end{document}